Els sistemes complexos es troben en nombrosos àmbits de la societat i la natura, des de l'organització cel·lular fins a l'estructura de les nostres ciutats. Són sistemes formats generalment per un gran nombre d'elements que interactuen entre si de forma no lineal, generant comportaments emergents, la qual cosa els converteix en sistemes difícils d'entendre i predir. Una forma habitual de caracteritzar els sistemes complexos ha estat buscar la relació entre l'estructura d'interaccions i el comportament col·lectiu, fixant una escala concreta o un cert tipus d'interaccions. L'estudi de la rica dinàmica dels sistemes complexos es pot veure beneficiada afrontant problemes des de perspectives diferents, combinant poc a poc més ingredients en la seva descripció. \\

Aquesta tesi explora dos tipus de sistemes complexos, ecològics i socials, i proposa un enfocament comú a tots dos. Concretament, s'aborden una sèrie de preguntes en relació a la coexistència i caracterització d'aquests sistemes des d'un punt de vista ecològic, tractant éssers vius, memes i agents en xarxes socials com a espècies que lluiten per la supervivència. Al llarg dels capítols de la tesi, desenvolupem models en diverses escales per entendre com diferents interaccions contribueixen a la persistència de les espècies. En el tercer capítol, l'ordenació espacial al nivell de l'individu és estudiada en sistemes purament competitius com a motor de coexistència i estabilitat. Demostrem que la dinàmica s'estabilitza quan els individus competeixen localment en un entorn estructurat. Augmentant l'ordre de descripció a espècies, en el quart capítol, remarquem la importància de combinar diversos tipus d'interaccions. Investiguem quines són les propietats de les espècies en la seva xarxa d'interaccions que determinen la supervivència quan considerem diverses interaccions alhora. Mitjançant machine learning, trobem que aquestes propietats canvien entre la situació en què la competició i el mutualisme són estudiats de forma aïllada i quan són considerats simultàniament. En aquest darrer cas, per predir la supervivència d'una espècie no només cal conèixer l'estructura de la xarxa ecològica, sinó també factors dinàmics com la intensitat de les interaccions en què participa. \\

La segona part de la tesi se centra en la descripció de xarxes socials sota un enfocament ecològic. Gràcies a l'existència de competició per l'atenció col·lectiva a les xarxes, utilitzem en el cinquè capítol l'analogia amb models ecològics per quantificar la competició i el mutualisme percebuda per usuaris i memes. Ens centrem en els canvis d'intensitat de les interaccions durant esdeveniments excepcionals per entendre com les xarxes socials hi responen, i saber anticipar-nos a situacions en les quals s'intenti manipular la salut dels nostres sistemes de comunicació. Els principals resultats de les nostres simulacions indiquen que durant els esdeveniments el mutualisme augmenta de tal manera que la competició neta entre usuaris disminueix. Aquest canvi també es reproduïx amb dades empíriques, corroborant que el mecanisme darrere de l'atenció col·lectiva és l'optimització per la visibilitat dels usuaris. Finalment, en el capítol sisè adoptem un enfocament macroscòpic per investigar si els omnipresents patrons d'abundància i diversitat que trobem en ecosistemes ecològics també estan presents en els ecosistemes socials. Trobem que la versió social d'aquests patrons existeix en bases de dades molt diverses, i que a més la seva forma funcional és similar a la dels ecològics. En conseqüència, els capítols reflecteixen els beneficis d'estudiar els sistemes socials sota una perspectiva ecològica, ja que models i teories pensats per a aquests últims sistemes poden també aplicar-se als primers. 