Los sistemas complejos se encuentran en numerosos ámbitos de la sociedad y la naturaleza, desde la organización celular hasta la estructura de nuestras ciudades. Son sistemas formados generalmente por un gran número de elementos que interactúan entre sí de forma no lineal generando comportamientos emergentes, lo que los convierte en sistemas difíciles de entender y predecir. Una forma habitual de caracterizar los sistemas complejos ha sido buscar la relación entre la estructura de interacciones y el comportamiento colectivo fijando una escala concreta o un cierto tipo de interacciones. El estudio de la rica dinámica de los sistemas complejos se puede ver beneficiada al afrontar problemas desde perspectivas diferentes, combinando poco a poco más ingredientes en su descripción. \\

Esta tesis explora dos tipos de sistemas complejos, ecológicos y sociales, y propone un enfoque común a ambos. En concreto, se abordan una serie de preguntas en relación a la coexistencia y caracterización de estos sistemas desde un punto de vista ecológico, tratando seres vivos, memes y agentes en redes sociales como especies que luchan por la supervivencia. A lo largo de los capítulos de la tesis, desarrollamos modelos en varias escalas para entender cómo diferentes interacciones contribuyen a la persistencia de las especies. En el tercer capítulo, la ordenación espacial al nivel del individuo es estudiada en sistemas puramente competitivos como motor de coexistencia y estabilidad. Demostramos que la dinámica se estabiliza cuando los individuos compiten localmente en un entorno estructurado. Aumentando el orden de descripción a especies, en el cuarto capítulo remarcamos la importancia de combinar varios tipos de interacciones. Investigamos cuales son las propiedades de las especies en su red de interacciones que determinan la supervivencia cuando consideramos varias interacciones a la vez. Mediante machine learning, encontramos que estas propiedades cambian entre la situación en la que la competición y el mutualismo son estudiados de forma aislada y cuando son considerados simultáneamente. En este último caso, para predecir la supervivencia de una especie no solo hace falta conocer la estructura de la red ecológica, sino también factores dinámicos como a intensidad de las interacciones en las que participa. \\

La segunda parte de la tesis se centra en la descripción de redes sociales bajo un enfoque ecológico. Gracias a la existencia de competición por la atención colectiva en las redes, utilizamos en el quinto capítulo la analogía con modelos ecológicos para cuantificar la competición y el mutualismo percibida por usuarios y memes. Nos centramos en los cambios de intensidad de las interacciones durante eventos excepcionales para entender cómo las redes sociales responden a ellos, y saber anteponernos a situaciones en las que se intente manipular la salud de nuestros sistemas de comunicación. Los principales resultados de nuestras simulaciones indican que durante los eventos el mutualismo aumenta de tal manera que la competición neta entre usuarios disminuye. Este cambio también se reproduce con datos empíricos, corroborando que el mecanismo detrás de la atención colectiva es la optimización por la visibilidad de los usuarios. Finalmente, en el capítulo sexto adoptamos un enfoque macroscópico para investigar si los omnipresentes patrones de abundancia y diversidad que encontramos en ecosistemas ecológicos también están presentes en los ecosistemas sociales. Encontramos que la versión social de estos patrones existe en muy diversas bases de datos, y que además su forma funcional es similar a la de los ecológicos. En consecuencia, los capítulos reflejan los beneficios de estudiar los sistemas sociales bajo una perspectiva ecológica, ya que modelos y teorías pensados para estos últimos sistemas pueden también aplicarse a los primeros.
\newpage

   