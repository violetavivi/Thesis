Me gustaría empezar dedicando unas palabras de agradecimiento a la gente que me ha acompañado durante estos años por todo el apoyo y el cariño. La aventura que es la tesis ha sido larga, y a veces tortuosa, pero también estimulante y enriquecedora gracias a las personas que he ido encontrando en el camino. Mis compañeros del IFISC (Javi, Jorge, Javier, Fer, Bea, María, Luiño, Emilio y muchos, muchos más), que han creado un gran ambiente en el instituto. El grupo de Padova, que me acogió con los brazos abiertos durante mi estancia. Mis colegas de la Santa Fe Summer School, que me reconectaron con mi amor por la ciencia. Y los investigadores que he conocido en congresos y escuelas. Gracias a todos por las conversaciones, discusiones científicas y momentos únicos. \\

En Mallorca he encontrado mi hogar durante este tiempo, pero también en Zaragoza he recibido el apoyo de mi familia, mi familia política, mis amigos, y el grupo con el nombre más origial del mundo: el Gotham lab. Especialmente, gracias a mi mejor amiga Vir por ser divertidísima y conocerme tan bien. \\

Durante este periodo de crecimiento tanto en el plano científico como en el personal, hay dos nombres propios que me han influenciado y ayudado muchísimo, sin los cuales esta tesis no hubiera llegado a ningún puerto. Carlos, gracias por ser el mejor compañero de aventuras y hacerme tan feliz. Sandro, mereces un agradecimiento especial.  Gracias inmensas por ser un gran ejemplo de persona y científico, y por esforzarte en transmitirme todos tus conocimientos. Has confiado en mí cuando ni yo misma lo hacía. Estas palabras son solo una fracción de lo que significas para mí \img{figures/appendices/heart-hands_1faf6.png}. \\

I also acknowledge the Spanish State Research Agency, through the Severo Ochoa and María de Maeztu Program for Centers and Units of Excellence in R\&D (Grant No. MDM2017-0711) \newline funded by MCIN/AEI/10.13039/501100011033, and through \newline Project No. PID2021-122256NB-C22 funded by \newline MCIN/AEI/10.13039/501100011033/FEDER, UE., the support from Grant No. FPI/2257/2019 Conv. 201 and CAIB Ph.D program.



