Complex systems are found in numerous areas of society and nature, from cellular organization to the structure of our cities. They are typically made up of a large number of elements that interact with each other in a nonlinear way generating emergent behaviors, which makes them difficult to understand and predict. A common way of characterizing complex systems has been to search for the relationship between the interaction structure and collective behavior, fixing a specific scale or type of interaction. The study of the rich dynamics of complex systems can benefit from approaching problems from different perspectives and gradually combining more ingredients in their description.\\

This thesis explores two types of complex systems, ecological and social, and proposes a common approach to both. Specifically, a series of questions are addressed regarding the coexistence and characterization of these systems from an ecological perspective, treating living beings, memes, and agents in social networks as species that struggle for survival. Throughout the thesis chapters, we develop models at various scales to understand how different interactions contribute to species persistence. In the third chapter, spatial ordering at the individual level is studied in purely competitive systems as a driver of coexistence and stability. We demonstrate that the dynamics stabilize when individuals compete locally in a structured environment. Increasing the description level to species, in the fourth chapter, we highlight the importance of combining several types of interactions. We investigate which properties of species in their interaction network determine survival when considering multiple interactions at once. Using machine learning, we find that these properties change between situations where competition and mutualism are studied in isolation and when they are considered simultaneously. In this latter case, to predict the survival of a species, it is not only necessary to know the structure of the ecological network, but also dynamic factors such as the intensity of the interactions in which it participates.\\

The second part of the thesis focuses on the description of social networks from an ecological perspective. Due to the existence of competition for collective attention in these networks, in the fifth chapter, we use an analogy with ecological models to quantify the competition and mutualism perceived by users and memes. We focus on changes in the intensity of interactions during exceptional events to understand how social networks respond to them and anticipate situations in which attempts may be made to manipulate the health of our communication systems. The main results of our simulations indicate that, during events, mutualism increases to such an extent that the net competition between users decreases. This change is also replicated with empirical data, corroborating that the mechanism behind collective attention is the optimization of user visibility. Finally, in the sixth chapter, we adopt a macroscopic approach to investigate whether the ubiquitous patterns of abundance and diversity that we find in ecological ecosystems are also present in social ecosystems. We find that the social version of these patterns exists in a variety of datasets and that their functional form is similar to that of ecological patterns. Consequently, the chapters reflect the benefits of studying social systems from an ecological perspective since models and theories designed for the latter systems can also be applied to the former.







