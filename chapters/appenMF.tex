\chapter{\label{appen:MF}Analytical formulation for structured interactions}

Along with a numerical implementation of the dynamics described in Chapter~\ref{chp1:1}, it is
also possible to provide a mathematical description of the
model, which we set up in this Appendix. In Section
\ref{sec:moments} we establish the basic equations for the
moments of the population variables. Section~\ref{sec:mf}
develops a standard mean-field approximation for statistically
homogeneous systems. We stress that it is not able to reproduce
the main numerical findings for our model, but gives a baseline
to interpret the results. Section~\ref{sec:localmf} extends the
mean-field approximation to allow for spatial inhomogeneity in
the species distribution. The results still do not match with
the numerical observations, but give some hints on the reduced
stability of homogeneous oscillations when the interaction
range is small.

\subsection{Moment equations}
\label{sec:moments}

An analytical description of the stochastic dynamics defined in
the main text can be given (after a trivial replacement of the
discrete-time dynamics by a continuous-time one) by the master
equation for the time-dependent probability of the system
state. It allows us to derive equations for the expected
relative abundance of each species at a given node as well as
for the two-node correlations.

The model state can be specified by giving $\{Z_\nu\}$, where
$Z_\nu=1,2,...,g$ specifies the species that occupies node
$\nu\in\Sigma$, with $\Sigma$ being the set of nodes of the
network. However, we find more convenient to parameterize the
model as follows. Let $n_{i,\nu}\in \{0,1\}$ be the number of
individuals of species $i\in\{1,\dots,g\}$ at node $\nu\in
\Sigma$, i.e. $n_{i,\nu}=1$ for one and only one $i$,
identifying the species present at $\nu$, and $0$ for the other
values of $i$ (absent species). The state of the system can be
characterized by the set of vectors $S=\{S_\nu\}_{\nu=1}^N$,
with $S_\nu=\{n_{1,\nu},\dots,n_{g,\nu}\}$. This state evolves
as follows: (i) with a rate $r$, a randomly chosen individual
(say, located at $\nu$) dies, then (ii) two neighbors of the
dead individual (thus pertaining to the set $P_\nu$ of
neighbors of $\nu$) are chosen at random and compete to
generate the offspring: a winner species is selected according
to the probabilities in the dominance matrix $H$. And (iii)
this offspring is immediately located at the vacant node.
Following standard procedures (for example see
 \cite{khlohe17,klkh20}) the master equation for the probability
$p(S,t)$ of finding the system in a state $S$ at time $t$ can
be written as
\begin{align}
  \label{eq:me}
  \frac{\partial}{\partial t}p(S,t)=
  \sum_{\nu=1}^N\sum_{i,j}\left( E^+_{i,\nu}E^-_{j,\nu}-1\right)\pi_\nu(i\to j)p(S,t),
\end{align}
where the operators $E^\pm$ act on an arbitrary state function
$f(S)$ as
\begin{eqnarray}
  \nonumber
E^\pm_{i,\nu}f(S)=f&\Big(&\{n_{1,1},\dots,n_{g,1}\} ,\dots, \\   \nonumber
&&\{n_{1,\nu},  \dots,n_{i,\nu}\pm 1,\dots,n_{g,\nu}\},\dots,\\
&&  \{n_{1,N},\dots,n_{g,N}\} \ \ \Big).
\end{eqnarray}
$\pi_{\nu}(i\to j)$ is the rate at which an individual of
species $i$ is replaced by one of species $j$ at site $\nu$,
given by
\begin{equation}
  \pi_{\nu}(i\to j)=r\frac{n_{i,\nu}}{N}\frac{2}{k_\nu(k_\nu-1)}
  \sum_{\stackrel{\lambda,\mu\in P_\nu}{\mu\neq \lambda}}\sum_{k}n_{j,\lambda}n_{k,\mu}H_{jk},
\end{equation}
where $k_\nu$ is the degree of node $\nu$, i.e. the number of
nodes in $P_\nu$. 


From the master equation we can derive equations for the
moments of the distribution, which can be easily measured from
the numerical simulations. The simplest nontrivial moment is
the expected number of individuals of species $i$ at node
$\nu$, $\mean{n_{i,\nu}}$. Its equation is readily obtained
from the master equation after multiplying it by $n_{i,\nu}$
and summing over all possible values of $S$:
\begin{eqnarray}
  \label{eq:ni}
  \frac{d}{ds} \mean{n_{i,\nu}} =&& \frac{1}{k_\nu(k_\nu-1)}
  \sum_{j} \sum_{\stackrel{\lambda,\mu\in P_\nu}{\mu\neq \lambda}}
  H_{ij} \mean{n_{i,\lambda}n_{j,\mu}}\nonumber \\
  && - \frac12 \mean{n_{i,\nu}} ,
\end{eqnarray}
where we have introduced a new time scale $s\equiv
\frac{2r}{N}t$.

From this equation we can write the dynamics for the expected
value of the macroscopic variable $x_i(s) \equiv N^{-1}\sum_\nu
n_{i,\nu}$ as 
\begin{equation}
\label{eq:xintermsofP}
  \frac{d}{ds} \mean{x_i(s)} =
  \sum_{j} H_{ij} P_{ij}(s) - \frac12 \mean{x_i(s)} ,
\end{equation}
where we have introduced the symmetric matrix
\begin{equation}
\label{eq:Ps}
  P_{ij}(s) = \frac{1}{N} \sum_\nu \frac{1}{k_\nu (k_\nu -1)}
  \sum_{\stackrel{\lambda,\mu\in P_\nu}{\mu\neq \lambda}} \mean{n_{i,\lambda}n_{j,\mu}} \ .
\end{equation}
This matrix can be interpreted as the probability of sampling
at time $s$ a pair of individuals of species $i$ and $j$ when
deciding the replacement of a dead individual somewhere in the
system. It satisfies  $\sum_{ij} P_{ij}(s) =1$
and, in a homogeneous network ($k_\nu=k$, $\forall \nu$),
$\sum_{j=1}^g P_{ij}(s)=\mean{x_i(s)}$.


As for the second-order moments, both for $\mu \in P_{\nu}$ and
for $\mu\notin P_\nu$, their equations read
\begin{eqnarray}
\nonumber
  \frac{d}{ds} \mean{n_{i,\nu}n_{j,\mu}} =&&\frac{1}{k_\nu(k_\nu-1)}
  \sum_{l}\sum_{\stackrel{\delta,\lambda\in P_\nu}{\delta\ne \lambda}}H_{il}\mean{n_{i,\lambda}n_{l,\delta}n_{j,\mu}} \\
  && + \frac{1}{k_\mu(k_\mu-1)}
  \sum_{l} \sum_{\stackrel{\delta,\lambda\in P_\nu}{\delta\ne \lambda}}H_{jl}\mean{n_{j,\lambda}n_{l,\delta}n_{i,\nu}} \nonumber \\ &&-\mean{n_{i,\nu}n_{j,\mu}}.
\label{eq:ninj}
\end{eqnarray}

In general, it can be seen that the moment equations form a
hierarchy, namely that the equation for a moment of order $o$
depends on the moments of order $o+1$. Hence, they cannot be
solved in closed form, except if introducing some
approximation.

\subsection{Homogeneous mean-field approximation}
\label{sec:mf}

The simplest of such approximations is the mean-field approach.
It is conveniently done in the simplified case in which the
network is spatially homogeneous, i.e. all the nodes have the
same degree: $k_\nu=k$, $\forall \nu$. In this situation, we
can search for statistically homogeneous solutions:
$\mean{n_{i,\nu}(s)}=\rho_i(s)$, $\forall \nu$. We can relate
these time-dependent moments $\rho_i(s)$ to the macroscopic
variables $x_i(s) \equiv N^{-1}\sum_\nu n_{i,\nu}(s)$ (note
that $\sum_{i=1}^g x_i(s)=1$). Indeed we have
$\mean{x_i}=N^{-1}\sum_\nu \rho_i=\rho_i$, or
$\mean{x_i}=\mean{n_{i,\nu}}$.

The mean-field approximation, which is exact in the case of
all-to-all interactions in an infinite system, and expected to
be accurate both for large enough interaction range (mean
degree) and for unstructured interactions, consists in
neglecting fluctuations and correlations:
\begin{eqnarray}
& \mean{n_{i,\nu}}=\mean{x_i}  \simeq x_i\ ,  & \forall \nu\in \Sigma, \\
& \mean{n_{i,\nu} n_{j,\mu}}   \simeq \mean{n_{i,\nu}} \mean{n_{j,\mu}} \simeq x_ix_j \ , &  \forall \nu\ne \mu \in \Sigma.
\end{eqnarray}
We have also $P_{ij}\approx x_i x_j$. Introduction of these
expressions into Eq. (\ref{eq:ni}) leads to a closed evolution
equation for $x_i$:
\begin{equation}
  \label{eq:meanf}
  \frac{d}{ds} x_i = \left(\sum_{l}H_{ij}x_j-\frac12\right)x_i \ .
\end{equation}

This mean-field equation has been studied before (e.g.
 \cite{Grilli2017Higher-orderModels}). We summarize here the
main results:

First, the dynamics (\ref{eq:meanf}) maintains in time the
property $\sum_i x_i =1$, if the initial condition satisfies
it. This can be seen by defining $X\equiv \sum_i x_i$,
calculating $dX/ds$, using that $H_{ij}=(H_{ij}+H_{ij})/2 =
(1-H_{ji}+H_{ij})/2$, and noticing that $\sum_{ij}
(H_{ij}-H_{ji})x_i x_j =0$ and $\sum_{ij} x_i x_j =X^2$. Thus
the sum of relative abundances satisfies
\begin{equation}
\label{eq:sumdynamics}
\frac{dX}{ds} = \frac12 (X^2 - X) \ ,
\end{equation}
which maintains $X(t)=1$, $\forall t$ if
$X(0)=1$.

Second, Eq. (\ref{eq:meanf}) has several equilibria or fixed
points. Many of them are of the `absorbing' or `boundary' type,
i.e. steady solutions of (\ref{eq:meanf}) in which $x_i=0$ for
some $i$, so that the corresponding species are extinct. In
addition, if $g$ is odd, there is generically
 \cite{Grilli2017Higher-orderModels} an \textit{interior}
equilibrium, $x_i(t)=x_i^*$, $\forall t$, in which all species
coexist with non-vanishing relative abundances $x_i^*$. At this
fixed point the relative abundances are given by
\begin{equation}
  \label{eq:sts}
  \sum_{j=1}^g H_{ij} x_j^* =\frac12 \ \ \Rightarrow \ \
  x_i^*=\frac12 \sum_j (H^{-1})_{ij},
\end{equation}
where $H^{-1}$ is the inverse of the dominance matrix, which
always exists when it describes an intransitive loop. The
properties of the boundary fixed points can be analyzed by
recognizing that they can be considered interior equilibria in
a system with a smaller number $g$ of species.

Third, the dynamics from arbitrary initial conditions in which
all $x_i$ are non-vanishing (and for generic $H$) leads to a
transient in which some of the species may become extinct. The
remaining ones, an odd number, cycle neutrally around the
interior fixed point (\ref{eq:sts}) in which the rows and
columns corresponding to the extinct species have been removed
from $H$ \cite{Grilli2017Higher-orderModels}. The stability of
this interior equilibrium is always neutral: relative
abundances of surviving species describe periodic closed orbits
around it, with an amplitude and period that is determined by
the initial condition and without being attracted nor repelled
by the fixed point. This can be seen
 \cite{Grilli2017Higher-orderModels} by noticing that the
quantity
\begin{equation}
V(x_1,...,x_g) = - \sum_{i=1}^g x_i^* \log \frac{x_i}{x_i^*}
\end{equation}
is a constant of motion, and thus it foliates the
$(g-1)$-simplex on which the dynamics occurs into invariant
hypersurfaces that turn out to contain concentric closed orbits
around the interior equilibrium.

The neutral character of the oscillations is not realistic from
the biological point of view, and structurally unstable from
the mathematical point of view. It is a consequence of the mean
field approximation, and we expect such neutral cycling to be
broken under corrections to mean-field, or under the full
dynamics with finite range of interaction. This is indeed what
is seen in our numerical simulations for the full model with
three species: either the fixed point becomes attracting, or
the neutral cycles are replaced by a single attracting limit
cycle, with amplitude and period independent of the initial
conditions.

In addition to its non-robust prediction of neutral cycling of
the species, the mean-field approximation is not able to
explain our main numerical finding: the fixed point
becomes stable for short-range interactions. From the
observations of Section~\ref{chp1:2.3} and Figure
\ref{chp1:fig:5} it is likely that the stabilization of the
fixed point arises from the fact that the relative abundance
$x_i$ are macroscopic quantities that become averaged and
non-fluctuating when the microscopic structure contains many
different domains, as in Figure \ref{chp1:fig:5}a. Thus,
it is pertinent trying to extend the mean-field formalism to
describe the microscopic spatially-dependent configurations, as
done in the following section.


\subsection{Local mean-field and spatial stability}
\label{sec:localmf}

In this section we consider the species locations to be at the
nodes of a two-dimensional square lattice. Then the node index
$\nu$ can be considered to be a discrete two-dimensional vector
$\nu$. For regular networks such as this one, the mean-field
approximation can be made local in space. This involves
removing correlations as
$\mean{n_{i,\mathbf{\nu}}n_{i,\nu}}\simeq\mean{n_{i,\nu}}\mean{n_{i,\nu}}$
while keeping the dependence of the mean quantities on the node
location.

Under this approximation, Eq.~\eqref{eq:ni} can be written as:

\begin{dmath}
 \frac{d}{ds}\rho_i(\nu,s) =  \frac{1}{k(k-1)} \sum_{j} H_{ij}  \left[
  \left(\sum_{\lambda\in P_\nu}\rho_i(\lambda,s)\right) \left(\sum_{\mu\in P_\nu}\rho_j(\mu,s)\right)
  - \sum_{\lambda \in P_\nu} \rho_i(\lambda,s) \rho_j(\lambda,s)
  \right]   -\frac{1}{2}\rho_i(\nu,s).
\label{eq:localMF}
\end{dmath}

We have used the notation $\mean{n_{i,\nu}} \equiv
\rho_i(\nu,s)$. Note that this equation reduces to Eq.
(\ref{eq:meanf}) when $\rho_i$ is homogeneous:
$\rho_i(\nu,s)=x_i(s)$, $\forall \nu$.

This new formulation allows us to assess the stability of
particular solutions against spatially-dependent perturbations.
For example we can focus on the stability of an homogeneous but
time-dependent solution $\rho_i(\nu,s)=x_i(s)$ which verifies
Eq.~\eqref{eq:meanf}. To do so, we seek a solution to
Eq.~\eqref{eq:localMF} of the form
\begin{equation}
  \rho_i(\nu,s)=x_i(s)+\delta_i(\nu,s),
\end{equation}
and linearize to first order in $\delta$. With this,
Eq.~\eqref{eq:localMF} becomes


\begin{dmath}
\frac{d\delta_i(\nu,s)}{ds} = \sum_j \frac{H_{ij}}{k}
\left[ x_j(s) \sum_{\lambda\in P_\nu} \delta_i(\lambda,s)
+  x_i(s) \sum_{\lambda\in P_\nu} \delta_j(\lambda,s)\right]  -\frac12 \delta_i(\nu,s)  \ .
\label{eq:localMFlin}
\end{dmath}

We introduce the Fourier transform of the perturbation:
$\hat\delta_i(q,s) = \sum_\nu e^{i q\cdot\nu}
\delta_i(\nu,s)$, in terms of which Eq.
(\ref{eq:localMFlin}) reads:
\begin{eqnarray}
  \frac{d\hat\delta_i(q,s)}{ds} &=& \left[ -1 + 2 F(q) \sum_j H_{ij} x_j(s) \right]\hat\delta_i(q,s) \nonumber \\
                                &+& F(q) x_i(s) \sum_j H_{ij} \hat\delta_j(q,s) \ .
\label{eq:stability}
\end{eqnarray}
We have introduced the quantity
\begin{equation}
  \label{eq:fourarA}
  F(q)\equiv \frac{1}{k}\sum_{\lambda \in P_{0}} e^{iq \cdot \lambda} \ ,
\end{equation}
which satisfies $F(q = 0)=1$, $|F(q)|\le 1$, and
$F(q)\to 0$ as $|q|\to \infty$. Note that this quantity
contains information on the interaction range through the
dependence on $P_{{0}}$ (i.e. through the set of neighbors of the origin). 
%{\VC DEFINE $P_{\bm{0}$}.

The simplest case to analyze is the stability of the interior
equilibrium point, i.e. $x_i(s)=x_i^*$, $\forall s$ as given by
Eq.~\eqref{eq:sts}. In this case Eq. \eqref{eq:stability} is a
linear system with constant coefficients, hence the stability
depends on the eigenvalues of the matrix of coefficients
$M_{ij}=F(q)x_i^* H_{ij} + [F(q)-1]\delta_{ij}/2$.  In
fact, because of Eq. (\ref{eq:sumdynamics}), there is always an
unstable eigenvalue $1/2$ for perturbations that bring the
dynamics out of the simplex. Thus, it is convenient to restrict
the dynamics to the simplex by using $\sum_{j=1}^g \delta_j=0$,
and then the matrix of the coefficients of Eq.
(\ref{eq:stability}) restricted to the first $g-1$ dimensions
is $M_{ij}=F(q)x_i^* (H_{ij}-H_{ig}) +
[F(q)-1]\delta_{ij}/2$, $i,j=1,...,g-1$.

For example, for $g=3$, the two eigenvalues of $M$ restricted
to the simplex can be explicitly calculated and read
\begin{equation}
  \lambda_\pm=-\frac{1-F}{2}
  \pm i \frac{F}{2}\sqrt{\frac{(2H_{12}-1)(2H_{13}-1)(2H_{23}-1)}{1 - 2(H_{12}-H_{13}+H_{23})}} \ .
\end{equation}
The argument of the square root is always positive when $H$
presents intransitive dominance cycles. Hence
\begin{equation}
  \textrm{Re}[\lambda_\pm]=-\frac{1-F(\mathbf q)}{2}\le 0
\end{equation}
and the equality holds if and only if $\mathbf q=\mathbf 0$.
This means that, within the mean-field approximation, the
steady and homogeneous solution $\rho_i(\nu,s)=x_i^*$ is 
linearly stable against small spatial perturbations, except for
homogeneous perturbations, in which stability is marginal (a
fact that we already knew from the more general nonlinear
arguments in Section~\ref{sec:mf}). Thus, the local mean-field
dynamics of Eq. (\ref{eq:localMF}) leads, for inhomogeneous
initial conditions close to the interior fixed point, to a
homogenization of the configuration, which then proceeds to
cycle neutrally around the fixed point. This is confirmed by
direct numerical simulation of Eq. (\ref{eq:localMF}). These
results hold for any value of the interaction range, contained
in $F(q)$. Thus, this local mean-field theory is not able to
explain the results from our stochastic model with structured
interactions. Namely, a transition from persistent inhomogeneous
configurations at short interaction range, which produce a
fully attracting fixed point for the macroscopic variable
$x_i(s)=\sum_\nu \rho(\nu,s)$, to a situation with
oscillatory dynamics that produces a repelling fixed point and
limit-cycle oscillations for $x_i(s)$ at large interaction
range.

Nevertheless, we can still use the local mean-field to gain
further insight into the dynamics, for example by analyzing the
stability with respect to inhomogeneous perturbations of a
homogeneous periodic solution $x_i(s)$ of Eq. (\ref{eq:meanf}).
In this case the stability equation (\ref{eq:stability}) is a
linear equation with periodic coefficients, which can be
analyzed with Floquet theory. The solutions can be written as a
linear combination of the functions \cite{gl94}
\begin{equation}
  f_i(s)e^{p_i s},\qquad i=1,\dots,g-1,
\end{equation}
where $f_i(s)$ are periodic (and hence bounded) functions of
time, with the same period $T$ as the functions $x_i(s)$, and
$p_i$ are the Floquet exponents given in terms of the
eigenvalues $\Lambda_i$ of the fundamental matrix $\Phi(s)$ of
system \eqref{eq:stability}, satisfying $\Phi(0)=I$, as
\begin{equation}
  \Lambda_i=e^{p_i T}.
\end{equation}
When all $p_i$ are negative, the perturbations decay and the
homogeneous solution $x_i(s)$ is recovered as time advances. We
have numerically evaluated $p_i$ for the case of three species,
$g=3$, and some values of the parameters of the system. For all
cases considered, $p_i$ has always negative real parts (except
for homogeneous perturbations, for which one finds neutral
stability), meaning that any initial inhomogeneous perturbation
tends to disappear. This agrees with direct simulation results
of Eq. (\ref{eq:localMF}). Thus, the long-term behavior of the local mean-field approach 
reduces to the standard homogeneous mean-field
treatment of Section~\ref{sec:mf}. In contrast, simulation of the
stochastic model shows domains of the different species for
short interaction range.

However, the stability strength is not the same for all
parameter values. Let $M(s)$ be the matrix of time-dependent
coefficients of the system \eqref{eq:stability}. A necessary,
but not sufficient, condition for the homogeneous solutions
$x_i(s)$ to be unstable is that some eigenvalue of $M(s)$ has
positive real part for some time $s\in[0,T]$  (see a proof of a
similar result in \cite{klkh20}). During these times, even if
the trajectory turns out to be linearly stable, its stability
is reduced and more susceptible to non-infinitesimal
perturbations or noise. For the case of three species $g=3$,
and the interaction matrix $H$ given again by Eq.~(\ref{eq:H}), we have seen
that the matrix $M(s)$ has eigenvalues with positive real
parts, for some possible periodic trajectories $x_i(s)$,
provided $F(q) \gtrsim 0.67$. Since the maximum of $F(q)$
occurs at zero wavenumber and the width of this function
decreases with increasing $k$, the band of wavenumbers
identified as `less-stable' shrinks as the interaction range,
quantified by $k$ increases. This is an indication (although
not a proof) that homogeneous periodic solutions would be more
robust for long-range interactions, and instabilities giving
rise to inhomogeneous configurations are more likely to occur
for short-range interactions. It is interesting to note that the
times at which the matrix $M(s)$ has more positive real-part of
eigenvalues coincide with the times at which some of the
components of the oscillatory solution $x_i(s)$ approach zero.


On general grounds, the local mean-field approximation should
represent some kind of coarse-graining of the original
stochastic system, and should be completed by noise terms to
gain accuracy. Under short-range interactions, appropriate
noise terms would be able to break the synchronization between
distant locations, and reproduce the domain structure
observed in the Monte Carlo simulations. However, we find
difficult to write analytical expressions for these noise terms
that would respect all the proper statistical constraints (for
example: reflect the multiplicative nature of birth-death
fluctuations, keep in time that $\sum_i \rho_i(\nu,s)=1$,
etc.). Also, the complexity of such model would not be lower
than the original individual-based one. Thus, we have not
developed further this possibility. 
%{\VC Las referencias se salen de la pagina}


\section{\label{sec:app_toymodel} Introducing correlations 
beyond mean-field}

In section \ref{chp1:2.3} we demonstrated that short-range interactions lead to the emergence of mono-specific clusters, effectively increasing intra-specific competition and 
stabilizing the dynamics.  As a further way to confirm that the decline of  inter-specific
competition is able to change the stability of the equilibrium,
making it stable for sufficiently reduced competition between
distinct species, we have studied a toy-model which shares
characteristics with our community model. It is built by
noticing that $\langle P_{ij} \rangle$ is just the time average
of the matrix $P_{ij}(s)$ in Eq. (\ref{eq:Ps}).
A way to correct the mean-field approximation $ P_{ij} \approx
x_i x_j$ is to introduce some correlations, $ P_{ij} \approx
c_{ij}x_i x_j$, making some ansatz for $c_{ij}$ and introducing it
into the exact equation (\ref{eq:xintermsofP}) (with
$\mean{x_i}\approx x_i$). We have explored the behavior of such
model in which correlations are implemented by
$c_{ij}=1-\epsilon$ if $i\neq j$ and $c_{ii}=1+\epsilon'$, with $\epsilon,\epsilon'>0$, 
resulting in an enhanced intra-specific competition with respect to
 inter-specific competition as $\epsilon$ and $\epsilon'$ are
increased. With this choice of $c_{ij}$ the resulting matrix
$P_{ij}$ does not have the proper statistical properties. In
particular the model does not respect that $\sum_i x_i=1$,
$\forall s$. However, this problem can be fixed by
constraining the dynamics onto the simplex by subtracting to
Eq. (\ref{eq:xintermsofP}), for each species $i$, the same term
$g^{-1}\sum_i G_i$, where $G_i$ is the right-hand-side of Eq.
(\ref{eq:xintermsofP}). It should be clear that this is not a
systematic approximation to our original system, but a toy
model useful to check the impact of varying the intra- and inter-specific competition balance. For example, for
$\epsilon'=0.01$ and the same dominance matrix used in Chapter~\ref{chp:1}, we have found that a Hopf bifurcation occurs at
$\epsilon = \epsilon_c\approx 0.01975$, so that relative
species abundances undergo limit cycle oscillations for
$\epsilon<\epsilon_c$ but the fixed point becomes stable and
attracting when the inter-specific competition is further
reduced, $\epsilon>\epsilon_c$. These are the same type of
states and the same transition that is encountered in our
stochastic model when decreasing the interaction range.
