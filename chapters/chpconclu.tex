\chapter{Conclusions}\label{chp:conclu}

In conclusion, we offer a comprehensive overview of our most significant findings and their implications. Our research has focused on the emergent behaviors that arise from the interactions among the components of complex systems, specifically ecological communities, and online social networks. Divided into two parts, the thesis examines each type of system, using a common approach that borrows concepts and models from ecology. In both parts, we study similar problems regarding coexistence but at different scales. We begin each part at the individual level by characterizing the effect of the spatial locations in Chapter~\ref{chp:1}, and quantifying interaction strengths in Chapter~\ref{chp:3}. We then increase our scale of description and study persistence at the species level in Chapter~\ref{chp:2}. In Chapter~\ref{chp:4}, we  deepen our abstraction to seek shared statistical patterns between social and ecological systems. Through the common ecological approach, we provide a unified framework for understanding the mechanisms that govern both ecological and information ecosystems. By bringing these systems together, we offer a more comprehensive understanding of the complex mechanisms that underlie their behavior. \\

Starting with the effect of spatial locations, Chapter~\ref{chp:1} models individuals competing in different spatial substrates. These spatial effects are thought to be one of the most prevalent mechanisms at play in real ecosystems \cite{Dieckmann2000}. Consequently, it is crucial to include space in the analysis of competitive communities, where stable coexistence cannot be achieved through intransitivity alone \cite{soliveres2018everything,godoy2017intransitivity,Grilli2017Higher-orderModels}. Our research suggests that local interactions promote stability by effectively reducing inter-species competition and damping large fluctuations in species abundances. In this sense, a limited interaction range cancels out the neutral oscillatory behavior characteristic of mean field models of intransitive interactions. Even though our results were derived from a simple model,  this Chapter adds a new perspective on how space can stabilize competitive communities. \\

While Chapter~\ref{chp:1} deals with interactions at the individual level, mechanisms of coexistence can also operate at the species level \cite{Grilli2017,Garcia-Callejas2018ThePersistence}. This is why we increase the scale of research in Chapter~\ref{chp:2} and look at the architecture of the species interactions. At the scale of species, an important question for persistence has been to explain why some species go extinct \cite{Sole2001, Keyes2021AnLosses}. We investigate how the network structure determines the probability of species survival after an environmental change when we consider multiple interaction types. The effect of network architecture has been extensively studied, but often focusing only on ecosystems' food webs and ignoring other interactions \cite{pascual2006ecological,kefi2020theoretical}. In this Chapter, we have taken into account two non-trophic interactions --competition and mutualism-- first in isolation and then simultaneously. We have addressed this challenge by simulating a dynamics embedded in the interaction network, and then by analyzing the structural properties of surviving species through machine learning. We have found that two node metrics that regard centrality are the main predictors of extinction. In mutualistic communities, species with low eigenvector centrality are more prone to extinction when all the interactions' strengths have been modified due to an environmental perturbation. In a competitive scenario, the same role is played by a high Page-Rank. However, when we model both interactions at the same time, predictors are not universal and vary with the structure of each community and the interaction strengths. The Chapter aimed to overcome some simplifications that have been used when modeling ecological interaction networks. Typically, only one type of interaction is usually addressed and, when the focus of the work is on structural analysis, the effect of the dynamics is neglected.  However, ecological communities are complex systems, and the results of studying multiple interactions together might not be the same as when they are studied separately \cite{kefi2012more,Kefi2015NetworkShores}. We have addressed mutualism and competition, but any other type of interaction, like predator-prey or parasitism, could be further included, together with  allometric-inspired dynamics, to improve realism. Nevertheless, as a consequence of the results of this work, we are now in a position to highlight the significance of revisiting classic results of interactions in isolation, which bring us closer to the ultimate goal of addressing ecological coexistence. \\

%Chapter 4:
%But networks are inherently difficult to understand, as
%the following list of possible complications illustrates...
%To make progress, different fields have suppressed certain
%complications while highlighting others. -> sim plified For instance, in nonlinear
%dynamics we have tended to favour simple, ...

% As a take-home message, when we study communities with both types of interactions, predictors are not universal and vary with the structure of each community and the interaction strengths. As a consequence of the results of this work, we are now in a position to highlight the significance of revisiting classic results of interactions in isolation, which bring us closer to the ultimate challenge of studying several interactions simultaneously to better understand ecosystems.

During the second part of the thesis, we maintain our pursuit of discovering mechanisms for coexistence, but we shift our focus to a slightly different system: online social networks. However, we do not abandon the ecological jargon and concepts from the first part. Instead, we leverage the similarities between ecological and social systems and approach the latter as information ecosystems. In this social context, 
we have addressed the drivers of collective attention, since it is crucial for understanding how to conserve the diversity and health of our information ecosystems \cite{palazzi2021ecological, gleeson2016effects}. In Chapter~\ref{chp:3},  we proposed a methodology to measure
the intensity of competitive and mutualistic interactions between users and hashtags based on Lotka-Volterra equations and niche theory. Our results show that during events that captivate collective attention, users experience reduced net competition. Then, the agreement between the
data and the numerical simulations suggests that the seek for visibility is what drives users’ behavior: users who participate in online debates tend to increase their visibility within the themes that interest them, creating intense competition inside each topic. The same visibility maximization encourages agents to follow the dominant topic as an event approaches, raising competition but also expanding viewers. This rearrangement of interactions is stable, however, because mutualistic benefits outweigh the more intense competition. \\

 Our results in Chapter~\ref{chp:3} shed light on one of the mechanisms behind the emergence of collective attention in online social media. The optimization of visibility and the consequences that we have found operate at the microscopic level, guiding the decisions of individual users. To unravel more mechanisms, in Chapter~\ref{chp:4} we have again scaled up our approach and studied the relationships between the agents of an information ecosystem, this time going to the broadest of perspectives, by characterizing and explaining statistical patterns of diversity and abundance. This task is also taken through an ecological perspective, to profit from the body of theory and philosophy that have been already developed in the study of macroecological patterns \cite{brown1995macroecology}. We have detected that the universality of ecological patterns found in both macroecological systems and microbiological communities is conserved in information ecosystems: i.e. even the same functional forms  of natural ecosystems are found in online social networks. Furthermore, as previously mentioned, identifying patterns is useful to construct theories and models, but equally valuable is identifying where these patterns are broken \cite{lego}. If a universal pattern is not present in a specific circumstance, it could indicate the presence of additional factors that are preventing the natural distribution from forming. This can have direct applications in information ecosystems, particularly when evaluating their health. For instance, a missing or distorted pattern may be the consequence of external interference, like bots disseminating misinformation. Understanding the general shape of these patterns may allow us to detect these threats more effectively, ultimately protecting the democratic discourse. As a result, one of the contributions of this thesis is in  fostering how an ecological perspective on the study of information ecosystems may offer practical resources for understanding social and collective processes.\\

 %This thesis has explored diverse aspects of the persistence of ecological communities and emergent phenomena in information ecosystems. Despite the approaches taken may seem pretty different --besides being inspired by ecology-- the chapters share a common viewpoint: they develop models to better explain, describe or predict empirical phenomena. While it is true that the various models emphasize different causal forces, this should be regarded as a benefit since their implications and insights overlap and interweave \cite{page2018model}. During this thesis, we have developed multiple models that operate at different scales for the same general goal. For example, we described a competitive dynamics through spatial networks of individuals in Chapter 3, and through a community network at the level of species in subsequent Chapter~\ref{chp:2} to study the emergence and maintenance of coexistence. The central idea behind our work is that by employing different models we can develop a nuanced and deep understanding to gain insights that may have otherwise been obscured. Models are just representations that can be expressed through mathematics and logic. All models share two common characteristics: simplification, which involves stripping away unnecessary details or abstracting from reality; and formalization, which uses precise definitions and mathematical expressions to represent the underlying concepts. By using different models, we emphasize some forces (space in Chapter~\ref{chp:1} or the species interactions in Chapter~\ref{chp:2}) while we simplify others (all species compete with  one another in Chapter~\ref{chp:1} and there are no spatial effects in Chapter ~\ref{chp:2}). Given that ecological --and complex-- systems function at multiple scales, it is only by collecting varied and even potentially contradictory sets of narratives that we can ultimately arrive at a more complete understanding. \\

In this thesis, we have explored diverse aspects of the persistence of ecological communities and emergent phenomena in information ecosystems. Even though we have developed different models, we can do it through the lens of complex systems. We have begun by using a sort of toy model in Chapter~\ref{chp:1}, which functions primarily as a mathematical representation, enabling us to state hypotheses and delve into their implications \textit{in sicilo}. As we progress through the thesis, we introduce more and more complexity and information by incorporating empirical data. Simulations are run on artificial and real network structures in Chapter~\ref{chp:2} to prove that theoretical results on systems with only one interaction type should be revisited. In Chapter ~\ref{chp:3}, a model inspired by Lotka-Volterra equations allows us to lay out the hypothesis that the driver of collective attention is users' visibility optimization, and its consequence is an increment in competition for attention. The importance of using data in this Chapter increases, since by using that model along with empirical data, we then are able to test and validate our hypothesis. The last Chapter of the thesis heavily relies on empirical data to identify patterns to provide a conceptual understanding of online social networks as information ecosystems. These patterns can be useful for generating hypotheses, inspiring models, or providing insights into complex phenomena that we would not have imagined looking at the raw data alone. As Dostoyevsky writes in \textit{Crime and Punishment}: ``We’ve got facts, they say. But facts aren’t everything; at least half the battle consists in how one makes use of them!''. Our patterns' description in Chapter~\ref{chp:4} reveals the statistical invariants that arise from the underlying drivers of information ecosystems (the facts) and paves the way for proposing new theories about their functioning. Given that ecological --and complex-- systems function at multiple scales, it is only by collecting a varied set of narratives through simulation and data that we can ultimately arrive at a more complete understanding.\\

%Furthermore, as mentioned earlier, not all models serve the same purposes. While the models of the first three Chapters can be regarded as more or less simple toy models, Chapter~\ref{chp:4} used a more descriptive approach. Toy models function primarily as a mathematical representation, enabling us to state hypotheses and delve into their implications in great detail. In Chapter ~\ref{chp:3}, for instance, a vitaminized model inspired by Lotka-Volterra equations allows us to lay out the hypothesis that the driver of collective attention is users' visibility optimization, and its consequence is an increment in competition for attention. By using the same model and empirical data, we then were able to test and validate our hypothesis. This empirical data is precisely what the other type of model employed in this thesis, the descriptive model, requires. The description of the patterns in Chapter~\ref{chp:4} aims to provide a conceptual understanding of online social networks as information ecosystems and relies on empirical data to identify those patterns. Descriptive models can then be useful for generating hypotheses or providing insights into complex phenomena that we would not have imagined looking at the raw data alone. As Dostoyevsky writes in \textit{Crime and Punishment}: ``We’ve got facts, they say. But facts aren’t everything; at least half the battle consists in how one makes use of them!''. Our patterns description in Chapter~\ref{chp:4} reveals the statistical invariants that arise from the underlying drivers of information ecosystems (the facts) and paves the way for proposing new theories about their functioning. \\

%Finally, we wish to emphasize the interdisciplinary nature of this thesis, which is evident in both the topics explored and the tools utilized. Our research addressed a range of subjects such as competition for space, ecological communities, and collective attention in social systems. To address these complex topics, we employed a variety of tools and techniques, like network theory, nonlinear dynamics, data analysis, and machine learning. All of these methods are linked by their ecological approach. Through our results, we shed light on how users' behaviors in social media are driven by information processing and, they highlight the opportunities that an ecological approach can present to the understanding of collective phenomena in social systems. In a broader sense, this thesis contributes to the integration of a complex-system philosophy in the study of both social and ecological systems. Our hope is that the examples presented here will inspire future research on ecology and computational social science with the foundations and applications of complex systems.

Finally, this thesis takes an interdisciplinary approach to exploring complex systems in both ecological and social settings. Through a range of subjects such as competition for space, survival in ecological communities, and collective attention in social systems, we employed a variety of tools and techniques, like network theory, nonlinear dynamics, data analysis, or machine learning. Our results shed light on how users' behaviors in social media are driven by information processing, and demonstrate the opportunities an ecological approach can present to the understanding of collective phenomena in social systems. The contributions of this thesis extend beyond our specific research and contribute to the integration of a complex-system philosophy in the study of both social and ecological systems. Our hope is that the examples presented here will inspire future research on ecology and computational social science with the foundations and applications of complex systems.

%lazer2020computational? 


 % TO CONCLU Our results throw light on how users' behaviors in social media are driven by information processing and, in a broader sense, they highlight the opportunities that an ecological approach can present to the understanding of collective phenomena in social systems. \\
 
