\chapter{Data and Code availability}\label{appen:DataCode}

 All Chapters of this thesis include computer simulations or statistical analysis, and Chapters~\ref{chp:2}, \ref{chp:3} and \ref{chp:4} rely on empirical data collected from real online social networks. This approach has proven useful for the prediction and characterization of the drivers behind social and ecological dynamics. However, as with any empirical approach, it depends on data. Nowadays, data is not usually freely available, but either often owned by private companies or ``available upon request''. This creates significant accessibility problems for research groups that lack sufficient funds to buy from private companies. If we advocate for open and reproducible science, this situation needs to be addressed urgently \cite{lazer2020computational, bartling2014opening}. In this thesis, we have done our bit by dedicating this Appendix to list the data sources (and code!) employed.\\

\paragraph{Chapter~\ref{chp:1}:} Code of the figures and some analysis of the work in which the chapter is based (\cite{calleja2022structured}) can be found in this repository: 
\href{https://github.com/violetavivi/Structured-interactions-as-a-stabilizing-mechanism-for-competitive-ecological-communities}{\texttt{https://github.com/violetavivi/Structured-interactions \newline -as-a-stabilizing-mechanism-for-competitive-ecological \newline -communities}} (accessed 20 January 2023).
%\url{shorturl.at/CJQR5}.

\paragraph{Chapter~\ref{chp:2}:} Code for simulating ecosystems and training decision trees can be found at \href{https://github.com/violetavivi/Structural-predictors}{\texttt{https://github.com/violetavivi \newline /Structural-predictors}} (accessed 22 February 2023).

\paragraph{Chapter~\ref{chp:3}:} All Twitter datasets are publicly available: Catalan self-determination referendum 2104 and Spanish  2019 national elections datasets are available at \url{https://osf.io/j5qwx/} (accessed 20 January 2023). For the Catalan self-determination referendum, data were collected by selecting all the tweets and new hashtags containing at least one hashtag from a list of around 70 predetermined hashtags or posted by one account from a list of approximately 50 users active in the referendum process and Catalan independence movements. Similar filtering was employed to collect the Spanish national elections dataset, where tweets and new hashtags were collected if containing at least one keyword from a set of 300 relevant terms that could either be Twitter usernames or hashtags related to the electoral process, such as candidates’ names, parties or campaign hashtags. The 2015 Nepal earthquake dataset can be downloaded from \href{https://figshare.com/articles/dataset/Twitter_event_datasets_2012-2016_/5100460}{\texttt{https://figshare.com/articles/dataset/ \newline Twitter\_event\_datasets\_2012-2016\_/5100460}} \newline (accessed 20 January 2023). In this case, the now-discontinued Twitter API was used with the filtering keywords: ‘nepal’, ‘earthquake’, ‘nepalearthquake’. The code of the abundance optimization model used in this chapter is an extended version from Palazzi \textit{et al.}  \cite{palazzi2021ecological}, which can be found at the COSIN3 group Github page: \newline
\href{https://github.com/COSIN3-UOC/dynamical-niche-model}{\texttt{https://github.com/COSIN3-UOC/dynamical-niche-model}} \newline(accessed 20 January 2023). 


\paragraph{Chapter~\ref{chp:4}:} The rest of the datasets are available at \href{https://figshare.com/articles/dataset/Twitter_event_datasets_2012-2016_/5100460}{\texttt{https:// figshare.com/articles/dataset/Twitter\_event\_datasets\_ \newline 2012-2016\_/5100460}} (accessed 20 January 2023). Similarly to the datasets of the previous Chapter, the tweets were collected using Twitter’s streaming API with a set of relevant hashtags and keywords related to each event,  which are publicly available in the aforementioned link. In turn, the abundance matrices and code to produce the Figures are stored at \href{https://github.com/violetavivi/Macroecological-patterns-in-online-social-networks}{\texttt{https://github.com/ \newline violetavivi/Macroecological-patterns-in-online-social- \newline networks}} (accessed 22 February 2023)